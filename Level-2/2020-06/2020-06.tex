\documentclass[11pt]{ctexart}

\usepackage{geometry}
\geometry{
    left = 0.6in,
    right = 0.6in,
    top = 0.8in,
    bottom = 1.0in
}
\usepackage{amssymb,amsbsy,amsmath,xcolor,mathrsfs,graphicx}
\usepackage{listings}
\usepackage{tasks}
\settasks{
    label = \Alph*. ,
    label-width = 16pt
}

\renewcommand{\title}[3]{
    \begin{center}
        \Large\heiti 中国电子学会 #1~年~#2~月 Python~#3级考试
    \end{center}
}
\newcommand{\TimeAndName}[1]{
    \begin{center}
        考试时间:~#1~ 分钟 \qquad\qquad\qquad\qquad 姓名:\underline{\quad\quad\quad\quad}
    \end{center}
}

\begin{document}
    \lstset{
        language = python,
        keywordstyle = \color{orange}\bfseries,
        emph = {
            abs, all, any, ascii, bin, bool, breakpoint, bytearray, bytes,
            callable, chr, classmethod, compile, complex, copyright, credits,
            delattr, dict, dir, divmod, enumerate, eval, exec, exit, filter,
            float, format, frozenset, getattr, globals, hasattr, hash,
            help, hex, id, input, int, isinstance, issubclass, iter, len,
            license, list, locals, map, max, memoryview, min, next, object,
            oct, open, ord, pow, print, property, quit, range, repr, reversed,
            round, set, setattr, slice, sorted, staticmethod, str, sum, super,
            tuple, type, vars, zip,
        },
        emphstyle = \color{purple}\bfseries,
        showspaces = false,
        basicstyle = \ttfamily,
        morekeywords = {True,False}
    }

    \title{2020}{6}{二}
    
    \TimeAndName{60}
    
    {\noindent\heiti 第一部分、单选题(共 25 题,每题 2 分,共50分.)}

    \begin{enumerate}
        % 1
        \item 下面的程序执行完毕后,最终的结果是?(\qquad)
        \begin{lstlisting}
            a = [34, 17, 7, 48, 10, 5]
            b = []
            c = []
            while len(a) > 0:
                s = a.pop()
                if s % 2 == 0:
                    b.append(s)
                else:
                    c.append(s)
            print(b)
            print(c)
        \end{lstlisting}
        \begin{tasks}(4)
            \task \lstinline{[34, 48, 10]}\\
            \lstinline{[17, 7, 5]}
            \task \lstinline{[10, 48, 34]}\\
            \lstinline{[5, 7, 17]}
            \task \lstinline{[10, 48, 34]}\\
            \lstinline{[17, 7, 5]}
            \task \lstinline{[34, 48, 10]}\\
            \lstinline{[5, 7, 17]}
        \end{tasks}

        % 2
        \item 以下程序的运行结果是?(\qquad)
        \begin{lstlisting}
        l = ["兰溪", "金华", "武义", "永康", "磐安", "东阳", "义乌", "浦江"]
        for s in l:
            if "义" in s:
                print(s, end=" ")
        \end{lstlisting}
        \begin{tasks}(4)
            \task 兰溪\ 金华\ 武义
            \task 武义\ 义乌
            \task 武义
            \task 义乌
        \end{tasks}

        % 3
        \item 以下程序的输出结果是?(\qquad)
        \begin{lstlisting}
            ls = [1,2,3]
            lt = [4,5,6]
            print(ls + lt)
        \end{lstlisting}
        \begin{tasks}(4)
            \task \lstinline{[1,2,3,4,5,6]}
            \task \lstinline{[1,2,3,[4,5,6]]}
            \task \lstinline{[4,5,6]}
            \task \lstinline{[5,7,9]}
        \end{tasks}

        % 4
        \item 列表 \lstinline{listV = list(range(10))},以下能够输出列表 \lstinline{listV} 中最小元素的是?(\qquad)
        \begin{tasks}(2)
            \task \lstinline{print(min(listV))}
            \task \lstinline{print(listV.max())}
            \task \lstinline{print(min(listV()))}
            \task \lstinline{print(listV.reverse(i)[0])}
        \end{tasks}

        % 5
        \item 以下程序的输出结果是?(\qquad)
        \begin{lstlisting}
            a = tuple('abcdefg')
            print(a)
        \end{lstlisting}
        \begin{tasks}(2)
            \task \lstinline{('a','b','c','d','e','f','g')}
            \task \lstinline{['a','b','c','d','e','f','g']}
            \task \lstinline{['abcdefg']}
            \task \lstinline{'abcdefg'}
        \end{tasks}

        % 6
        \item 运行如下程序,结果是?(\qquad)
        \begin{lstlisting}
            l = [1, "laowang", 3.14, "laoli"]
            l[0] = 2
            del l[1]
            print(l)
        \end{lstlisting}
        \begin{tasks}(2)
            \task \lstinline{[1,3.14,'laoli']}
            \task \lstinline{[2,3.14,'laoli']}
            \task \lstinline{["laowang",3.14,'laoli']}
            \task \lstinline{[2,"laowang",3.14]}
        \end{tasks}

        % 7
        \item 关于列表 \lstinline{s} 的相关操作,描述不正确的是?(\qquad)
        \begin{tasks}
            \task \lstinline{s.append()}:在列表的末尾添加新的对象
            \task \lstinline{s.reverse()}:反转列表中的元素
            \task \lstinline{s.count()}:统计某个元素在列表中出现的次数
            \task \lstinline{s.clear()}:删除列表 \lstinline{s} 的最后一个元素
        \end{tasks}

        % 8
        \item 关于以下代码,描述正确的是?(\qquad)
        \begin{lstlisting}
            a = 'False'
            if a:
                print('True')
        \end{lstlisting}
        \begin{tasks}(2)
            \task 上述代码的输出结果为 True
            \task 上述代码的输出结果为 False
            \task 上述代码存在语法错误
            \task 上述代码没有语法错误,但没有任何输出
        \end{tasks}

        % 9
        \item 以下代码的输出结果是?(\qquad)
        \begin{lstlisting}
            ls = [[0,1],[5,6],[7,8]]
            lis = []
            for i in range(len(ls)):
                lis.append(ls[i][1])
            print(lis)
        \end{lstlisting}
        \begin{tasks}(4)
            \task \lstinline{[1,6,8]}
            \task \lstinline{[0,5,7]}
            \task \lstinline{[0,6,8]}
            \task \lstinline{[0,1]}
        \end{tasks}

        \newpage
        % 10
        \item 已知列表 \lstinline{lis = ['1','2',3]},则执行 \lstinline{print(2 in lis)} 语句输出的结果是?(\qquad)
        \begin{tasks}(4)
            \task \lstinline{True}
            \task \lstinline{true}
            \task \lstinline{False}
            \task \lstinline{false}
        \end{tasks}

        % 11
        \item 现在有 \lstinline{s = 'abcdefghi'},请问 \lstinline{s[4]} 的值是?(\qquad)
        \begin{tasks}(4)
            \task \lstinline{d}
            \task \lstinline{e}
            \task \lstinline{abcd}
            \task \lstinline{0}
        \end{tasks}

        % 12
        \item 下面代码的输出结果是?(\qquad)
        \begin{lstlisting}
            a = {'sx':90, 'yuwen':93, 'yingyu':88, 'kexue':98}
            print(a['sx'])
        \end{lstlisting}
        \begin{tasks}(4)
            \task 93
            \task 90
            \task 88
            \task 98
        \end{tasks}

        % 13
        \item 下面代码的输出结果是?(\qquad)
        \begin{lstlisting}
            a = [1,3,5,7,9]
            for i in a:
                print(i, end=" ")
        \end{lstlisting}
        \begin{tasks}(4)
            \task \lstinline{1,3,5,7,9}
            \task \lstinline{[1,3,5,7,9]}
            \task \lstinline{1 3 5 7 9}
            \task \lstinline{9 7 5 3 1}
        \end{tasks}

        % 14
        \item 以下用于 Python 循环结构的关键字是?(\qquad)
        \begin{tasks}(4)
            \task \lstinline{while}
            \task \lstinline{loop}
            \task \lstinline{if}
            \task \lstinline{do...for}
        \end{tasks}
        
        % 15
        \item 以下代码绘制的图形是?(\qquad)
        \begin{lstlisting}
            import turtle
            for i in range(1, 7):
                turtle.fd(50)
                turtle.left(60)
        \end{lstlisting}
        \begin{tasks}(4)
            \task 正方形
            \task 六边形
            \task 三角形
            \task 五角星
        \end{tasks}

        % 16
        \item 已知列表 \lstinline{a = [1,2,3]},\lstinline{b = ['4']},执行语句 \lstinline{print(a + b)} 后,输出的结果是?(\qquad)
        \begin{tasks}(2)
            \task \lstinline{[1,2,3,4]}
            \task \lstinline{[1,2,3,'4']}
            \task \lstinline{['1','2','3','4']}
            \task \lstinline{10}
        \end{tasks}

        % 17
        \item 已知列表 \lstinline{a = [1,2,3,4,5]},下列语句输出结果为 \lstinline{False} 的是?(\qquad)
        \begin{tasks}(2)
            \task \lstinline{print(a[3] == a[-2])}
            \task \lstinline{print(a[:3] == a[:-2])}
            \task \lstinline{print(a[3:] == a[-2:])}
            \task \lstinline{print(a[2] == a[-3])}
        \end{tasks}

        % 18
        \item 在 Python 中,表示跳出当前循环的语句是?(\qquad)
        \begin{tasks}(4)
            \task \lstinline{break}
            \task \lstinline{pass}
            \task \lstinline{exit}
            \task \lstinline{Esc}
        \end{tasks}

        % 19
        \item 以下程序执行的结果是?(\qquad)
        \begin{lstlisting}
            score = {"语文":95, "数学":93, "英语":97}
            print(socre["语文"] + score["数学"] // 2)
        \end{lstlisting}
        \begin{tasks}(4)
            \task 141
            \task 141.5
            \task 94
            \task 94.0
        \end{tasks}

    %     \newpage
        % 20
        \item 下面代码的结果是?(\qquad)
        \begin{lstlisting}
            a = {"name":"jt", "age":29, "class":5}
            a["age"] = 15
            a["school"] = "派森社"
            print("age:", a["age"])
            print("school", a["school"])
        \end{lstlisting}
        \begin{tasks}(4)
            \task 9
            \task \lstinline{'a + 4'}
            \task 无结果,出错
            \task \lstinline{a + 4}
        \end{tasks}

        % 21
        \item 下列程序的执行结果是?(\qquad)
        \begin{lstlisting}
            s = (1, 2, 3, 4, 5, 6, 7, 8)
            print(len(s), max(s), min(s))
        \end{lstlisting}
        \begin{tasks}(4)
            \task 7 8 1
            \task 8 8 1
            \task 8 1 8
            \task 7 1 8
        \end{tasks}  

        % 22
        \item 下列不属于 Python 中处理字典的方法是?(\qquad)
        \begin{tasks}(4)
            \task \lstinline{pop()}
            \task \lstinline{replace()}
            \task \lstinline{get()}
            \task \lstinline{popitem()}
        \end{tasks}

        % 23
        \item 下列语句,不能创建元组的是?(\qquad)
        \begin{tasks}(4)
            \task \lstinline{tup=()}
            \task \lstinline{tup=(1)}
            \task \lstinline{tup=1,2}
            \task \lstinline{tup=(1,2)}
        \end{tasks}  

        % 24
        \item \lstinline{s = 'abc123'},采用字符串操作函数将其中的字符 \lstinline{c} 替换为字符 \lstinline{C},以下哪个操作正确?(\qquad)
        \begin{tasks}(2)
            \task \lstinline{s.replace('c', 'C')}
            \task \lstinline{replace(c, C)}
            \task \lstinline{s.replace(c, C)}
            \task \lstinline{replace('abc123', 'abC123')}
        \end{tasks}

        % 25
        \item 下面程序的执行结果为?(\qquad)
        \begin{lstlisting}
            s = '{0} + {1} + {2}'.format(2, 3, 5)
            print(s)
        \end{lstlisting}
        \begin{tasks}(4)
            \task $0+1=2$
            \task $\{0\}+\{1\}=\{2\}$
            \task $2+3=5$
            \task $\{2\}+\{3\}=\{5\}$
        \end{tasks}
    \end{enumerate}

    \newpage
    {\noindent\heiti 第二部分、判断题(共 10 题,每题 2 分,共20分.)}
    \begin{enumerate}
        \setcounter{enumi}{25}
        % 26
        \item 已知字符串 \lstinline{string = "www.baidu.com"},那么 \lstinline{print(string.split('.', 1))} 的结果是:\\\lstinline{['www', 'baidu', 'com']}(\qquad)

        %27
        \item 运行如下程序的结果为 \lstinline{hellohelloeverybody}(\qquad)
        \begin{lstlisting}
            a1 = 'hello'
            a2 = 'everybody'
            print(a1 * 2 + a2)
        \end{lstlisting}
        
        %28
        \item 元组中不可以通过下标索引获取元素(\qquad)
  
        %29
        \item 在使用 \lstinline{get()} 语句返回字典中指定键的值的时候,如果该键的值在字典中不存在,则返回系统默认值“\lstinline{unknown}”(\qquad)
        
        %30
        \item 元组是一种可变的序列,创建后可以修改(\qquad)
        
        %31
        \item 条件语句中,\lstinline{if} 语句和 \lstinline{if...else} 语句没有区别(\qquad)
        
        %32
        \item 列表是一种序列,列表的元素可以追加、替换、插入和删除(\qquad)
        
        %33
        \item 在循环语句中,\lstinline{break} 语句的作用是提前进入下一次循环(\qquad)
        
        %34
        \item 执行以下程序,运行的结果是:14(\qquad)
        \begin{lstlisting}
            a = 1
            while a < 100:
                if a % 2 == 0 and a % 7 == 0:
                    print(a)
                    break
                a = a + 1
        \end{lstlisting}
        
        %35
        \item 下列代码执行后输出的结果为 \lstinline{0,1,2,}(\qquad)
        \begin{lstlisting}
            for i in range(3):
                print(i, end=",")
        \end{lstlisting}
    \end{enumerate}

    \newpage
    {\noindent\heiti 第三部分、编程题(共 2 题,共30分.)}
    \begin{enumerate}
        \setcounter{enumi}{35}
        
        % 36
        \item 数字转汉字:
        \begin{tasks}[label = (\arabic*)]
            \task 提示用户输入一个 $1\sim 9$ 之间的整数;
            \task 根据用户输入,输出对应的汉字;
            \task 可以重复输入。
        \end{tasks}
        【输入样例:】
        
        1

        【输出样例:】

        一
        \vfill

        %37
        \item 打分:
        
        小明去参加歌唱比赛,现在给出 10 位评委的打分,要求实现如下功能:
        \begin{tasks}[label = (\arabic*)]
            \task 去掉一个最高分和一个最低分后计算平均分(平均分取整数);
            \task 然后按照以下格式显示出来。
        \end{tasks}
        【输入样例:】

        99 80 86 89 94 92 75 87 86 95

        【输出样例:】

        去掉一个最高分:99分,去掉一个最低分:75,最后得分为:88
        \vfill
    \end{enumerate}
\end{document}