\documentclass[11pt]{ctexart}

\usepackage{geometry}
\geometry{
    left = 0.6in,
    right = 0.6in,
    top = 0.8in,
    bottom = 1.0in
}
\usepackage{amssymb,amsbsy,amsmath,xcolor,mathrsfs,graphicx}
\usepackage{listings}
\usepackage{tasks}
\settasks{
    label = \Alph*. ,
    label-width = 16pt
}
\pagestyle{empty}

\renewcommand{\title}[3]{
    \begin{center}
        \Large\heiti 中国电子学会 #1~年~#2~月 Python~#3级考试
    \end{center}
}
\newcommand{\TimeAndName}[1]{
    \begin{center}
        考试时间:~#1~ 分钟 \qquad\qquad\qquad\qquad 姓名:\underline{\quad\quad\quad\quad}
    \end{center}
}

\begin{document}
    \lstset{
        language = python,
        keywordstyle = \color{orange}\bfseries,
        emph = {
            abs, all, any, ascii, bin, bool, breakpoint, bytearray, bytes,
            callable, chr, classmethod, compile, complex, copyright, credits,
            delattr, dict, dir, divmod, enumerate, eval, exec, exit, filter,
            float, format, frozenset, getattr, globals, hasattr, hash,
            help, hex, id, input, int, isinstance, issubclass, iter, len,
            license, list, locals, map, max, memoryview, min, next, object,
            oct, open, ord, pow, print, property, quit, range, repr, reversed,
            round, set, setattr, slice, sorted, staticmethod, str, sum, super,
            tuple, type, vars, zip,
        },
        emphstyle = \color{purple}\bfseries,
        showspaces = false,
        basicstyle = \ttfamily,
        morekeywords = {True,False}
    }

    \title{2021}{3}{二}
    
    \TimeAndName{60}
    
    {\noindent\heiti 第一部分、单选题(共 25 题,每题 2 分,共50分.)}

    \begin{enumerate}
        % 1
        \item 对于字典 \lstinline!infor = {"name":"tom", "age":13, "sex":"male"}!,删除 \lstinline!"age":13! 键值对的操作正确的是?(\qquad)
        \begin{tasks}(2)
            \task \lstinline!del infor['age']!
            \task \lstinline!del infor['age':13]!
            \task \lstinline!del infor!
            \task \lstinline!infor.clear()!
        \end{tasks}

        % 2
        \item 用Python语句计算:长方形的长和宽分别为4和5,则长方形的周长为?(\qquad)
        \begin{tasks}(4)
            \task \lstinline!a=4!\\\lstinline!b=5!\\\lstinline!c=2*a+2*b!\\\lstinline!print(c)!
            \task \lstinline!a=4!\\\lstinline!b=5!\\\lstinline!c=2*a+b!\\\lstinline!print(c)!
            \task \lstinline!a=4!\\\lstinline!b=5!\\\lstinline!c=a*b!\\\lstinline!print(c)!
            \task \lstinline!a=4!\\\lstinline!b=5!\\\lstinline!c=a+b*2!\\\lstinline!print(c)!
        \end{tasks}

        % 3
        \item 下列代码输出结果是?(\qquad)
        \begin{lstlisting}
            ist1 = ['A','&','A',8,'A']
            list1.remove('A')
            print(list1)
        \end{lstlisting}
        \begin{tasks}(4)
            \task \lstinline!['A','&','A',8]!
            \task \lstinline!['&','A','8',A]!
            \task \lstinline!['&',8]!
            \task \lstinline!['A','&',8,'A']!
        \end{tasks}

        % 4
        \item 数学课代表将同学们的数学成绩存放在列表 \lstinline{s1} 中,\lstinline!s1=[99,92,87,90,100,95]!,如果按照成绩由低到高输出,以下哪个程序可以实现?(\qquad)
        \begin{tasks}(2)
            \task \lstinline!s1 = [99,92,87,90,100,95]!\\\lstinline!s2 = sorted(s1)!\\\lstinline!print(s2)!
            \task \lstinline!s1 = [99,92,87,90,100,95]!\\\lstinline!s2 = sort()!\\\lstinline!print(s2)!
            \task \lstinline!s1 = [99,92,87,90,100,95]!\\\lstinline!sort(reverse=True)!\\\lstinline!print(s1)!
            \task \lstinline!s1 = [99,92,87,90,100,95]!\\\lstinline!s2 = sorted(s1, reverse=True)!\\\lstinline!print(s2)!
        \end{tasks}

        % 5
        \item  执行下列语句,将输出?(\qquad)
        \begin{lstlisting}
            >>> list1=['b','c',1,2,3,4,5] 
            >>> list1.append('a') 
            >>> list1
        \end{lstlisting}
        \begin{tasks}(2)
            \task \lstinline!['b','c',1,2,3,4,5,'a']!
            \task 无任何输出
            \task \lstinline{b}
            \task \lstinline!['b','c',1,2,3,4,5]!
        \end{tasks}

        \newpage
        % 6
        \item 已知 \lstinline!t=(88,77,95,64,85)!,那么 \lstinline!t[1:3]! 的结果是?(\qquad)
        \begin{tasks}(4)
            \task \lstinline![88,77]!
            \task \lstinline!(88,77)!
            \task \lstinline!(77,95)!
            \task \lstinline![77,95]!
        \end{tasks}

        % 7
        \item 我们可以定义一个字典 \lstinline{week1},用数字 $1\sim 7$ 表示中文的星期一到星期日(如1表示星期一),正确的语句是?(\qquad)
        \begin{tasks}
            \task \footnotesize \lstinline!week1 = (1:"星期一",2:"星期二",3:"星期三",4:"星期四",5:"星期五",6:"星期六",7:"星期日")!
            \task \footnotesize \lstinline!week1 = [1:"星期一",2:"星期二",3:"星期三",4:"星期四",5:"星期五",6:"星期六",7:"星期日"]!
            \task \footnotesize \lstinline!week1 = \{'1':"星期一",'2':"星期二",'3':"星期三",'4':"星期四",'5':"星期五",'6':"星期六",'7':"星期日"\}!
            \task \footnotesize \lstinline!week1 = \{1:"星期一",2:"星期二",3:"星期三",4:"星期四",5:"星期五",6:"星期六",7:"星期日"\}!
        \end{tasks}

        % 8
        \item  下列不会产生死循环的程序是?(\qquad)
        \begin{tasks}(2)
            \task \lstinline{i = 1}\\
            \lstinline{while True:}\\
            \lstinline{\ \ \ \ i += 1}\\
            \lstinline{\ \ \ \ if i \% 2 == 0:}\\
            \lstinline{\ \ \ \ \ \ \ \ continue}\\
            \lstinline{\ \ \ \ print(i)}

            \task \lstinline{i = 1}\\
            \lstinline{while True:}\\
            \lstinline{\ \ \ \ if i \% 2 == 0:}\\
            \lstinline{\ \ \ \ \ \ \ \ continue}\\
            \lstinline{\ \ \ \ print(i)}

            \task \lstinline{i = 0}\\
            \lstinline{while True:}\\
            \lstinline{\ \ \ \ i += 1}\\
            \lstinline{\ \ \ \ if i == 100:}\\
            \lstinline{\ \ \ \ \ \ \ \ break}\\
            \lstinline{\ \ \ \ print(i)}

            \task \lstinline{i = 1}\\
            \lstinline{while i == 1:}\\
            \lstinline{\ \ \ \ print(i)}
        \end{tasks}

        % 9
        \item 下列程序运行完成时,\lstinline{i} 的值为?(\qquad)
        \begin{lstlisting}
            for i in "I enjoy coding.":
                if i == "c":
                    break
                print(i, end="")
        \end{lstlisting}
        \begin{tasks}(4)
            \task \lstinline!'o'!
            \task \lstinline!'c'!
            \task \lstinline!"I enjoy"!
            \task \lstinline!''!
        \end{tasks}

        % 10
        \item  语句 \lstinline!list6=[0,6]! 的含义是?(\qquad)
        \begin{tasks}(2)
            \task 定义一个变量 \lstinline{list6},值为 $0\sim 6$ 之间的随机值
            \task 定义一个变量 \lstinline{list6},值为0.6
            \task 定义一个列表 \lstinline{list6},包含两个元素:0和6
            \task 生成一个数字序列 \lstinline{list6},值为0到6
        \end{tasks}

        \newpage
        % 11
        \item 有如下Python程序段:
        \begin{lstlisting}
            x = 10
            y = 5
            if x/y == x//y:    
                print("相等")
            else:    
                print("不相等")
        \end{lstlisting}
        执行程序段后,输出的结果是?(\qquad)
        \begin{tasks}(4)
            \task \lstinline{"相等"}
            \task \lstinline{"不相等"}
            \task \lstinline!相等!
            \task \lstinline!不相等!
        \end{tasks}

        % 12
        \item  下列代码的输出结果是?(\qquad)
        \begin{lstlisting}
            lis=list(range(5))
            print(lis)
        \end{lstlisting}
        \begin{tasks}(4)
            \task \lstinline!0,1,2,3,4,5!
            \task \lstinline![0,1,2,3,4]!
            \task \lstinline!0,1,2,3,4!
            \task \lstinline![0,1,2,3,4,5]!
        \end{tasks}

        % 13
        \item  执行下列语句,将输出?(\qquad)
        \begin{lstlisting}
            >>> x = [1,2,3,4,5]
            >>> y = [2,3,5]
            >>> z.append(x[1]*y[2])
            >>> z
        \end{lstlisting}
        \begin{tasks}(4)
            \task \lstinline!x[1]*y[2]!
            \task \lstinline!25!
            \task \lstinline![3]!
            \task \lstinline![10]!
        \end{tasks}

        % 14
        \item 下列程序的运行结果是?(\qquad)
        \begin{lstlisting}
            a = 1
            if a > 0:
                a = a + 1
            if a > 1:
                a = 5
            print(a)
        \end{lstlisting}
        \begin{tasks}(4)
            \task 1
            \task 2
            \task 5
            \task 0
        \end{tasks}
        
        % 15
        \item 已知字符串 \lstinline!st='Python'!,执行语句 \lstinline!x=a[::2]! 后,变量 \lstinline{x} 的值为?(\qquad)
        \begin{tasks}(4)
            \task \lstinline!'Pyt'!
            \task \lstinline!'Py'!
            \task \lstinline!'yhn'!
            \task \lstinline!'Pto'!
        \end{tasks}

        % 16
        \item 已知\lstinline!a="1"!,\lstinline!b="2"!,则表达式 \lstinline{a + b} 的值为?(\qquad)
        \begin{tasks}(4)
            \task 3
            \task 12
            \task '12'
            \task '21'
        \end{tasks}

        % 17
        \item 运行如下代码后的值为?(\qquad)
        \begin{lstlisting}
            list1 = [1,3,4]
            list2 = [3,5,2]
            (list1 + list2)*2
        \end{lstlisting}
        \begin{tasks}(2)
            \task \lstinline![8,16,12]!
            \task \lstinline![1,3,4,3,5,2,1,3,4,3,5,2]!
            \task \lstinline![1,3,4,1,3,4,3,5,2,3,5,2]!
            \task \lstinline![2,6,8,6,10,4]!
        \end{tasks}

        % 18
        \item 下列代码输出结果是?(\qquad)
        \begin{lstlisting}
            ls=['python','2021']
            print(type(ls))
        \end{lstlisting}
        \begin{tasks}(4)
            \task \lstinline!<class 'dict'>!
            \task \lstinline!<class 'set'>!
            \task \lstinline!<class 'list'>!
            \task \lstinline!<class 'tuple'>!
        \end{tasks}

        % 19
        \item 已知 \lstinline!t = (2,3,5,7,9)!,下列哪条指令可以求元组数据的和?(\qquad)
        \begin{tasks}(4)
            \task \lstinline!len(t)!
            \task \lstinline!min(t)!
            \task \lstinline!max(t)!
            \task \lstinline!sum(t)!
        \end{tasks}

        % 20
        \item 已知变量\lstinline!stra="IloveTX"!,执行语句 \lstinline!print("love" in stra")! 的结果为?(\qquad)
        \begin{tasks}(4)
            \task \lstinline!True!
            \task \lstinline!False!
            \task 1
            \task 0
        \end{tasks}

        % 21
        \item 有如下 Python 程序段:
        \begin{lstlisting}
            lista = [1,2,3,4,5,6,7,8,9,10]
            s = 0
            for i in range(0,len(lista),2):    
                s = s+lista[i]
            print("s=",s)
        \end{lstlisting}
        执行程序段后,输出的结果为?(\qquad)
        \begin{tasks}(4)
            \task 25
            \task \lstinline!s=25!
            \task \lstinline!s=30!
            \task \lstinline!s=55!
        \end{tasks}

        \newpage
        % 22
        \item 运行下列程序语句后,字典 \lstinline{a} 是空值的是?(\qquad)
        \begin{tasks}
            \task \lstinline!a = \{'职业':'警察','年龄':25:'姓名':'李四'\}!\\
                  \lstinline!a.clear()!
            \task \lstinline!a = \{'职业':'警察','年龄':25:'姓名':'李四'\}!\\
                  \lstinline!c = \{'职业':'教师'\}!\\       
                  \lstinline!a.clear()!
            \task \lstinline!a = \{'职业':'警察','年龄':25:'姓名':'李四'\}!\\
                  \lstinline!del a['职业']!
            \task \lstinline!a = \{'职业':'警察','年龄':25:'姓名':'李四'\}!\\
                  \lstinline!a.popitem()!
        \end{tasks}

        % 23
        \item 下列程序的运行结果是?(\qquad)
        \begin{lstlisting}
            L = [1,2,3,4,5,2,1]
            L.pop(3)
            L.pop(2)
            print(L)
        \end{lstlisting}
        \begin{tasks}(4)
            \task \lstinline![1,2,4,5,1]!
            \task \lstinline![1,2,3,5,2]!
            \task \lstinline![1,2,5,2,1]!
            \task \lstinline![1,2,3,5,1]!
        \end{tasks}

        % 24
        \item 有如下Python程序段:
        \begin{lstlisting}
            tup1 = (1,2,3,4,5)
            x = tup1[1]+tup1[-1]
            print("x=",x)
        \end{lstlisting}
        执行上述程序段后,输出的结果为?(\qquad)
        \begin{tasks}(4)
            \task 3
            \task \lstinline!x=3!
            \task 7
            \task \lstinline!x=7!
        \end{tasks}

        % 25
        \item 下列程序运行的结果是?(\qquad)
        \begin{lstlisting}
            infor = {"name":"tom", "age":13, "sex":"male"}
            print(len(infor))
        \end{lstlisting}
        \begin{tasks}(4)
            \task 6
            \task 3
            \task 4
            \task 1
        \end{tasks}
    \end{enumerate}

    \newpage
    {\noindent\heiti 第二部分、判断题(共 10 题,每题 2 分,共20分.)}
    \begin{enumerate}
        \setcounter{enumi}{25}
        % 26
        \item 下列程序的输出结果是\lstinline!('A','p','p,'l','e')!.(\qquad)
        \begin{lstlisting}
            vowels = ('a','p','p,'l','e')
            vowels[0] = 'A'
            print(vowels)
        \end{lstlisting}

        %27
        \item Python语句的循环结构中 \lstinline{for} 循环是条件循环.(\qquad)
        
        %28
        \item 字典中的元素称为键值对,包括一个键和一个值,键和值中间用逗号隔开.(\qquad)
  
        %29
        \item 在循环语句中 \lstinline{break} 语句的作用是提前结束所有循环.(\qquad)
        
        %30
        \item 若 \lstinline!s='春眠不觉晓,处处闻啼鸟。'!,则 \lstinline!s[2:4]! 的值是 \lstinline{'不觉'}.(\qquad)
        
        %31
        \item 元组是可变的,可以通过下标索引访问元素.(\qquad)
        
        %32
        \item \lstinline!"好好学习" + "天天向上"! 的输出结果是 \lstinline!"好好学习""天天向上"!.(\qquad)
        
        %33
        \item 当使用循环时,有时候在满足某个条件时,想要退出循环,我们可以使用 \lstinline!break! 语句.(\qquad)
        
        %34
        \item 程序设计的三种基本结构为:顺序结构、选择结构和分支结构(\qquad)
        
        %35
        \item \lstinline!count()! 方法用于统计某个元素在列表中出现的次数.(\qquad)
    \end{enumerate}

    {\noindent\heiti 第三部分、编程题(共 2 题,共30分.)}
    \begin{enumerate}
        \setcounter{enumi}{35}
        
        % 36
        \item 设计一个停车场收费计算器(收费规则,2小时以内收费5元,超出部分每小时加收2元),要求如下:
        
        \begin{tasks}[label=(\arabic*)]
            \task 设计的程序要能输入停车时间(单位为小时,输入的小时数为整数)
            \task 程序可以根据输入的停车时间自动计算出停车费,并且显示出来
            \task 程序可以重复使用
        \end{tasks}
        \vfill

        %37
        \item 用户输入一个半径r,求该半径下的圆的面积s与周长c。要求如下
        
        \begin{tasks}[label=\arabic*.]
            \task 输出的面积与周长都保留俩位小数
            \task 输出的格式为:\lstinline!“圆的周长是**,面积是**”!
            \task $\pi$取3.14
            \task 使用 \lstinline!print()! 格式化输出(\% 方法)
        \end{tasks}
        \vfill
    \end{enumerate}
\end{document}