\documentclass[11pt]{ctexart}

\usepackage{geometry}
\geometry{
    left = 0.6in,
    right = 0.6in,
    top = 0.8in,
    bottom = 1.0in
}
\usepackage{amssymb,amsbsy,amsmath,xcolor,mathrsfs,graphicx}
\usepackage{listings}
\usepackage{tasks}
\settasks{
    label = \Alph*. ,
    label-width = 16pt
}
\pagestyle{empty}

\renewcommand{\title}[3]{
    \begin{center}
        \Large\heiti 中国电子学会 #1~年~#2~月 Python~#3级考试
    \end{center}
}
\newcommand{\TimeAndName}[1]{
    \begin{center}
        考试时间:~#1~ 分钟 \qquad\qquad\qquad\qquad 姓名:\underline{\quad\quad\quad\quad}
    \end{center}
}

\begin{document}
    \lstset{
        language = python,
        keywordstyle = \color{orange}\bfseries,
        emph = {
            abs, all, any, ascii, bin, bool, breakpoint, bytearray, bytes,
            callable, chr, classmethod, compile, complex, copyright, credits,
            delattr, dict, dir, divmod, enumerate, eval, exec, exit, filter,
            float, format, frozenset, getattr, globals, hasattr, hash,
            help, hex, id, input, int, isinstance, issubclass, iter, len,
            license, list, locals, map, max, memoryview, min, next, object,
            oct, open, ord, pow, print, property, quit, range, repr, reversed,
            round, set, setattr, slice, sorted, staticmethod, str, sum, super,
            tuple, type, vars, zip,
        },
        emphstyle = \color{purple}\bfseries,
        showspaces = false,
        basicstyle = \ttfamily,
        morekeywords = {True,False}
    }

    \title{2020}{12}{二}
    
    \TimeAndName{60}
    
    {\noindent\heiti 第一部分、单选题(共 25 题,每题 2 分,共50分.)}

    \begin{enumerate}
        % 1
        \item 执行以下代码中,输出的结果是?(\qquad)
        \begin{lstlisting}
            s = 0
            for i in range(1,10,3):    
                s = s + i
                print(s, i)
        \end{lstlisting}
        \begin{tasks}(4)
            \task 22 10
            \task 12 7
            \task 45 9
            \task 55 10
        \end{tasks}

        % 2
        \item 已知\lstinline!s=list("sgdhasdghasdg")!,以下选项中能输出字符"\lstinline{g}"出现的次数的是?(\qquad)
        \begin{tasks}(2)
            \task \lstinline{print(s.index(g))}
            \task \lstinline{print(s.index("g"))}
            \task \lstinline{print(s.count("g"))}
            \task \lstinline{print(s.count(g))}
        \end{tasks}

        % 3
        \item 下列代码的执行结果是?(\qquad)
        \begin{lstlisting}
            s1 = "abcde"
            s2 = "fgh"
            s3 = s1+s2
            s3[4:7]
        \end{lstlisting}
        \begin{tasks}(4)
            \task efg
            \task efgh
            \task def
            \task defg
        \end{tasks}

        % 4
        \item 以下代码的输出结果是?(\qquad)
        \begin{lstlisting}
            ls1 = [1,2,3,4,5]
            ls2 = ls1 
            ls2.reverse()
            print(ls1)
        \end{lstlisting}
        \begin{tasks}(4)
            \task \lstinline!5,4,3,2,1!
            \task \lstinline![1,2,3,4,5]!
            \task \lstinline![5,4,3,2,1]!
            \task \lstinline!1,2,3,4,5!
        \end{tasks}

        % 5
        \item  运行如下代码,结果是?(\qquad)
        \begin{lstlisting}
            l = ["a",1,"b",[1,2]]
            print(len(l))
        \end{lstlisting}
        \begin{tasks}(4)
            \task 3
            \task 4
            \task 5
            \task 6
        \end{tasks}

        % 6
        \item 以下代码的输出结果是?(\qquad)
        \begin{lstlisting}
            lis = list(range(4))
            print(lis)
        \end{lstlisting}
        \begin{tasks}(4)
            \task \lstinline![0,1,2,3,4]!
            \task \lstinline![0,1,2,3]!
            \task \lstinline!0,1,2,3!
            \task \lstinline!0,1,2,3,4!
        \end{tasks}

        % 7
        \item 下面代码的输出结果是?(\qquad)
        \begin{lstlisting}
            ls = ["橘子","芒果","草莓","西瓜","水蜜桃"]
            for k in ls:    
                print(k,end=" ")
        \end{lstlisting}
        \begin{tasks}(2)
            \task \lstinline!橘子芒果草莓西瓜水蜜桃!
            \task \lstinline!橘子\ 芒果\ 草莓\ 西瓜\ 水蜜桃!
            \task \lstinline!西瓜!
            \task \lstinline!"橘子""芒果""草莓""西瓜""水蜜桃"!
        \end{tasks}

        % 8
        \item  关于 Python 中的流程控制语句,下列描述错误的是?(\qquad)
        \begin{tasks}
            \task 在分支结构中,\lstinline{if}、\lstinline{elif}、\lstinline{else} 都可以单独使用
            \task 分支结构中的条件判断通常用关系表达式或逻辑表达式来进行描述
            \task \lstinline{continue} 语句和 \lstinline{break} 语句只有在循环结构中才可以使用
            \task \lstinline{while} 语句和 \lstinline{for} 语句的循环条件后都必须输入冒号,需要循环执行的语句必须有缩进
        \end{tasks}

        % 9
        \item 已知有列表 \lstinline{a = [1,2,3,4,5]},以下语句中,不能输出 \lstinline{[5,4,3,2,1]} 的是?(\qquad)
        \begin{tasks}(2)
            \task \lstinline!print(a[:-6:-1])!
            \task \lstinline!print(a.sort(reverse=True))!
            \task \lstinline!print(sorted(a, reverse=True))!
            \task \lstinline!print([5,4,3,2,1])!
        \end{tasks}

        % 10
        \item 已知列表 \lstinline{a=[1,2,3,4,5]}, 执行 \lstinline{a.insert(2,6)} 后结果是什么?(\qquad)
        \begin{tasks}(4)
            \task \lstinline![1,2,3,4,5,2,6]!
            \task \lstinline![1,2,3,4,5,6]!
            \task \lstinline![1,2,6,3,4,5]!
            \task \lstinline![1,2,3,6,4,5]!
        \end{tasks}

        % 11
        \item 下列选项中,不属于 Python 流程控制语句的是?(\qquad)
        \begin{tasks}(4)
            \task if-elif-else语句
            \task while语句
            \task do-while语句
            \task for语句
        \end{tasks}

        % 12
        \item 关于 Python 元组类型,以下选项中描述错误的是?(\qquad)
        \begin{tasks}
            \task 元组不可以被修改
            \task Python 中元组使用圆括号和逗号表示
            \task 元组中的元素要求是相同类型
            \task 一个元组可以作为另一个元组的元素,可以采用多级索引获取信息
        \end{tasks}

        % 13
        \item 设有元组\lstinline!tup=(1,2,3,'1','2','3')!,执行语句\lstinline!print(tup[0::2])!,得到的结果是?(\qquad)
        \begin{tasks}(4)
            \task \lstinline!(1,2)!
            \task \lstinline!(1,3)!
            \task \lstinline!(1,3,'2')!
            \task \lstinline!(1,'1','3')!
        \end{tasks}

        % 14
        \item  执行下列代码,输出的结果是?(\qquad)
        \begin{lstlisting}
            dic = {'a': 1, 'b': 2, 'c': 3, 'd': 4}
            dic.pop('b')
            del dic['d']
            dic['d'] = 4
            print(dic)
        \end{lstlisting}
        \begin{tasks}(2)
            \task \lstinline!\{'a':1,'b':2,'c':3,'d':4\}!
            \task \lstinline!\{'a':1,'c':3,'d':4\}!
            \task \lstinline!\{'a':1,'c':3\}!
            \task \lstinline!\{'d':4\}!
        \end{tasks}
        
        % 15
        \item 已知列表\lstinline!a=[1, 2,'3']!,执行语句\lstinline!print(a*2)!后,输出的结果是?(\qquad)
        \begin{tasks}(4)
            \task \lstinline![1,2,'3',1,2,'3']!
            \task \lstinline![1,2,'3']!
            \task \lstinline![2,4,'6']!
            \task \lstinline![1,2,'3',2]!
        \end{tasks}

        % 16
        \item 下列代码的运行结果是?(\qquad)
        \begin{lstlisting}
            a={'xm':'zhangsan'}
            b={'sg':175,'tz':'55kg'}
            b.update(a)
            len(b)
        \end{lstlisting}
        \begin{tasks}(4)
            \task 2
            \task 3
            \task 4
            \task 6
        \end{tasks}

        % 17
        \item 下面Python循环体执行的次数与其他不同的是?(\qquad)
        \begin{tasks}(2)
            \task \lstinline{i = 0}\\
            \lstinline{while i <= 10:}\\
            \lstinline{\ \ \ \ print(i)}\\
            \lstinline{\ \ \ \ i = i + 1}
            \task \lstinline{i = 10}\\
            \lstinline{while i > 0:}\\
            \lstinline{\ \ \ \ print(i)}\\
            \lstinline{\ \ \ \ i = i - 1}
            \task \lstinline{for i in range(10):}\\
            \lstinline{\ \ \ \ print(i)}\\
            \task \lstinline{for i in range(10, 0, -1):}\\
            \lstinline{\ \ \ \ print(i)}\\
        \end{tasks}

        % 18
        \item 下列属于列表的是?(\qquad)
        \begin{tasks}(2)
            \task \lstinline!str1 = "python"!
            \task \lstinline!list1 = ['1','2','3']!
            \task \lstinline!tup1 = ('1','2','3')!
            \task \lstinline!dict1 = \{'a':1,'b':2,'c':3\}!
        \end{tasks}

        % 19
        \item 已知\lstinline!t=(1,2,3,4,5,6)!,下面哪条元组操作是非法的??(\qquad)
        \begin{tasks}(4)
            \task \lstinline!len(t)!
            \task \lstinline!max(t)!
            \task \lstinline!min(t)!
            \task \lstinline!t[1]=8!
        \end{tasks}

        % 20
        \item 已知字符串中的某个字符,要找到这个字符的位置,最简便的方法是?(\qquad)
        \begin{tasks}(4)
            \task 切片
            \task 连接
            \task 分割
            \task 索引
        \end{tasks}

        % 21
        \item 以下关于字典特性的描述正确的是?(\qquad)
        \begin{tasks}
            \task 字典支持位置索引
            \task 字典是一种有序的对象集合
            \task 字典中的数据可以进行切片
            \task 字典里面的值(不是键)可以包含列表和其他数据类型
        \end{tasks}

        % 22
        \item 在某学校,张三、李四、王五三名同学对应的学号分别是100、101、102。现将他们学号与姓名对应的关系存入字典 \lstinline{id_name} 中,关于以下代码段的描述,错误的是?(\qquad)
        \begin{lstlisting}
            id_name = {101: '张三', 102: '李四', 103: '王五'}
            name_id = {}
            name_id[id_name[101]] = 101
            name_id[id_name[102]] = 102
            name_id[id_name[103]] = 103
            print(id_name == name_id)
        \end{lstlisting}
        \begin{tasks}
            \task 在字典 \lstinline{id_name} 中,只能通过学号查找对应的姓名,不能通过姓名查找对应的学号
            \task 上述代码是将 \lstinline{id_name} 中姓名作为键、学号作为值构建了一个新的字典 \lstinline{name_id}
            \task 上述代码执行后,字典 \lstinline{name_id} 为: \lstinline!\{'张三': 101, '李四': 102, '王五': 103\}!
            \task 上述代码输出的结果为: \lstinline!True!
        \end{tasks}

        % 23
        \item 执行下面代码,结果是?(\qquad)
        \begin{lstlisting}
            for key in "lanxi":
                if key == "x":        
                    break    
                print(key)
        \end{lstlisting}
        \begin{tasks}(4)
            \task lanxi
            \task lan
            \task l\\a\\n
            \task l\\a\\n\\i
        \end{tasks}

        % 24
        \item 关于 \lstinline{break} 语句与 \lstinline{continue} 语句的说法中,以下选项中错误的是?(\qquad)
        \begin{tasks}
            \task \lstinline{continue} 语句类似于 \lstinline{break} 语句,也必须在 \lstinline{for}、\lstinline{while} 循环中使用
            \task \lstinline{break} 语句结束循环,继续执行循环语句的后续语句
            \task 当多个循环语句嵌套时,\lstinline{break} 语句只适用于当前嵌套层的语句
            \task \lstinline{continue} 语句结束循环,继续执行循环语句的后续语句
        \end{tasks}

        % 25
        \item 对\lstinline!s="www.baidu.com"!执行\lstinline!s.split(".")!后的结果是什么?(\qquad)
        \begin{tasks}(2)
            \task \lstinline!www.baidu.com!
            \task \lstinline!['www','baidu','com']!
            \task \lstinline!"www.baidu.com"!
            \task \lstinline!wwwbaiducom!
        \end{tasks}
    \end{enumerate}

    {\noindent\heiti 第二部分、判断题(共 10 题,每题 2 分,共20分.)}
    \begin{enumerate}
        \setcounter{enumi}{25}
        % 26
        \item 执行以下代码,输入数字99,运行结果是:ok(\qquad)
        \begin{lstlisting}
            a=input('输入一个数字:') 
            if a<100:     
                print('ok')
        \end{lstlisting}

        %27
        \item 在 Python 中 \lstinline!for item in range(1,10,2)!表示的是从1循环到10(包括10)步长是2(\qquad)
        
        %28
        \item \lstinline!d=()!,\lstinline{d} 是一个空列表.(\qquad)
  
        %29
        \item 在使用 \lstinline{del} 语句删除字典中不需要的元素时,必须指定字典名和要删除的键.(\qquad)
        
        %30
        \item 判断下面的语句是否正确.(\qquad)
        \begin{lstlisting}
            >>> a="Hello"-"World"
            >>> a
            >>> 'Hello World'
        \end{lstlisting}
        
        %31
        \item 在Python中\lstinline!range()!,表示一个整数序列,对于浮点型和字符串类型是无效的.(\qquad)
        
        %32
        \item 元组的访问速度比列表要快一些,如果定义了一系列常量值,并且主要用途仅仅是对其进行遍历而不需要进行任何修改,建议使用元组而不使用列表.(\qquad)
        
        %33
        \item 元组是用方括号来表示的,列表是用圆括号来表示的.(\qquad)
        
        %34
        \item 运行如下代码的结果为 \lstinline!"今天是3月25日,星期三,天气晴好"!.(\qquad)
        \begin{lstlisting}
            print('今天是%d月%d日,星期%s,天气%s'%(3,25,'三','晴好'))
        \end{lstlisting}
        
        %35
        \item \lstinline{for} 循环适合已知循环次数的操作,\lstinline{while} 循环适合未知循环次数的操作.(\qquad)
    \end{enumerate}

    {\noindent\heiti 第三部分、编程题(共 2 题,共30分.)}
    \begin{enumerate}
        \setcounter{enumi}{35}
        
        % 36
        \item 成绩等级:(10分)
        
        编写一段代码,要求如下:
        \begin{tasks}[label=\arabic*.]
            \task 程序开始运行后,需要用户输入学生的成绩(成绩为正整数)
            \task 一次输入一个学生的成绩,学生成绩是从0到100
            \task 根据用户输入的成绩,程序依据等级标准,输出相应的等级
            \task 等级标准是成绩小于60为不及格,60(含)到85(不含)之间为良,85(含)以上为优
            \task 可以重复输入成绩进行查询
        \end{tasks}
        \vfill

        %37
        \item 剔除数字:
        
        要求如下:
        \begin{tasks}[label=\arabic*.]
            \task 编写一段程序代码,程序运行后,需要用户随意输入一段包含有数字和字母的字符串
            \task 程序会自动删除字符串中的数字,然后输出一串没有数字的字符串(纯字母的字符串)或者列表(没有数字)
            \task 要求输出的非数字的字符顺序不能变
        \end{tasks}
        \vfill
    \end{enumerate}
\end{document}