\documentclass[11pt]{ctexart}

\usepackage{geometry}
\geometry{
    left = 0.6in,
    right = 0.6in,
    top = 0.8in,
    bottom = 1.0in
}
\usepackage{amssymb,amsbsy,amsmath,xcolor,mathrsfs,graphicx}
\usepackage{listings}
\usepackage{tasks}
\usepackage{pifont}
\settasks{
    label = \Alph*. ,
    label-width = 16pt
}

\renewcommand{\title}[3]{
    \begin{center}
        \Large\heiti 中国电子学会 #1~年~#2~月 Python~#3级考试
    \end{center}
}
\newcommand{\TimeAndName}[1]{
    \begin{center}
        考试时间:~#1~ 分钟 \qquad\qquad\qquad\qquad 姓名:\underline{\quad\quad\quad\quad}
    \end{center}
}

\begin{document}
    \lstset{
        language = python,
        keywordstyle = \color{orange}\bfseries,
        emph = {
            abs, all, any, ascii, bin, bool, breakpoint, bytearray, bytes,
            callable, chr, classmethod, compile, complex, copyright, credits,
            delattr, dict, dir, divmod, enumerate, eval, exec, exit, filter,
            float, format, frozenset, getattr, globals, hasattr, hash,
            help, hex, id, input, int, isinstance, issubclass, iter, len,
            license, list, locals, map, max, memoryview, min, next, object,
            oct, open, ord, pow, print, property, quit, range, repr, reversed,
            round, set, setattr, slice, sorted, staticmethod, str, sum, super,
            tuple, type, vars, zip,
        },
        emphstyle = \color{purple}\bfseries,
        showspaces = false,
        basicstyle = \ttfamily,
        morekeywords = {True,False}
    }

    \title{2021}{6}{一}
    
    \TimeAndName{60}
    
    {\noindent\heiti 第一部分、单选题(共 25 题,每题 2 分,共50分.)}

    \begin{enumerate}
        % 1
        \item 下列程序运行的结果是?(\qquad)
        \begin{lstlisting}
            s = 'hello'
            print(s + 'world')
        \end{lstlisting}
        \begin{tasks}(4)
            \task \lstinline{sworld}
            \task \lstinline{helloworld}
            \task \lstinline{hello}
            \task \lstinline{world}
        \end{tasks}
    
        % 2
        \item 下列选项中不符合Python语言变量命名规则的是?(\qquad)
        \begin{tasks}(4)
            \task \lstinline{Computer}
            \task \lstinline{P}
            \task \lstinline{3\_1}
            \task \lstinline{\_WO1}
        \end{tasks}

        % 3
        \item 在 Python 中,运行 \lstinline!9//2! ,输出的结果是?(\qquad)
        \begin{tasks}(4)
            \task $3$
            \task $4.5$
            \task $4$
            \task $4.0$
        \end{tasks}
    
        % 4
        \item 下面哪一行代码的输出结果不是 \lstinline{World2021} ?(\qquad)
        \begin{tasks}
            \task \lstinline!print("World" + "2021")!
            \task \lstinline!print("World" + "20"+"21")!
            \task \lstinline!print("World" + 2021)!
            \task \lstinline!print("World2021")!
        \end{tasks}

        % 5
        \item 在Python中,输入 \lstinline!3 * 4 ** 2! ,运算结果是?(\qquad)
        \begin{tasks}(4)
            \task 144
            \task 24
            \task 48
            \task 6
        \end{tasks}

        % 6
        \item 关于比较运算符说法正确的是?(\qquad)
        
        \ding{172} \lstinline{!=} 表示为不等于,如果两个操作数不相等,则为 \lstinline{False}

        \ding{173} \lstinline{<=} 表示为小于等于,如果左边的数小于或等于右边的数,则为 \lstinline{True}

        \ding{174} 若\lstinline{a = 2}, \lstinline{b = 5} 则 \lstinline{a != b} 为 \lstinline{True}
        \begin{tasks}(4)
            \task \ding{172}\ding{173}
            \task \ding{173}\ding{174}
            \task \ding{172}\ding{174}
            \task \ding{172}\ding{173}\ding{174}
        \end{tasks}

        % 7
        \item Python中的乘法是用哪个符号表示的?(\qquad)
        \begin{tasks}(4)
            \task \lstinline!*!
            \task \lstinline!X!
            \task \lstinline!x!
            \task \lstinline!\#!
        \end{tasks}

        % 8
        \item 以下哪个选项可以作为Python文件的后缀名?(\qquad)
        \begin{tasks}(4)
            \task .py
            \task .png
            \task .doc
            \task .pdf
        \end{tasks}

        % 9
        \item 要给三个整型变量a、b、c赋值为5,下面Python程序正确的是?(\qquad)
        \begin{tasks}(4)
            \task \lstinline!abc=5!
            \task \lstinline{a=5,b=5,c=5}
            \task \lstinline!a=b=c=5!
            \task \lstinline!a=5 b=5 c=5!
        \end{tasks}

        % 10
        \item 以下哪段程序能在画出三角形并隐藏turtle?(\qquad)
        \begin{tasks}(2)
            \task 
            \lstinline!import turtle! \\ 
            \lstinline!turtle.circle(150, steps=3)! \\
            \lstinline!turtle.hideturtle()! \\
            \lstinline!turtle.done()!

            \task 
            \lstinline!import turtle! \\ 
            \lstinline!turtle.circle(150, 3)! \\
            \lstinline!turtle.hideturtle()! \\
            \lstinline!turtle.done()!

            \task 
            \lstinline!import turtle! \\ 
            \lstinline!turtle.circle(3)! \\
            \lstinline!turtle.hideturtle()! \\
            \lstinline!turtle.done()!

            \task \lstinline!import turtle! \\ 
            \lstinline!turtle.circle(150,3,3)! \\
            \lstinline!turtle.hideturtle()! \\
            \lstinline!turtle.done()!
        \end{tasks}

        % 11
        \item \lstinline{turtle.home()} 的作用是下列哪一种?(\qquad)
        \begin{tasks}(2)
            \task 移至初始坐标 (0,0)
            \task 移至初始坐标 (0,0),并设置朝向为初始方向
            \task 移至屏幕左上角
            \task 设置朝向为初始方向
        \end{tasks}

        % 12
        \item 关于Turtle绘图,下列说法错误的是?(\qquad)
        \begin{tasks}
            \task 色彩处理时,可以使用彩色画笔 \lstinline{pencolor()},也可以直接由 \lstinline{color()} 方法更改目前画笔的颜色
            \task \lstinline{penup()} 指的是将笔提起,不会绘制任何图形
            \task 在选择画笔粗细时可以使用 \lstinline{pensize()}
            \task 在海龟绘图中,画布中央是(0,0),往右$x$坐标值递减,往左$x$坐标值递增
        \end{tasks}

        % 13
        \item  在Python中,输入 \lstinline!18 / 6 // 3!,输出结果为?(\qquad)
        \begin{tasks}(4)
            \task 1
            \task 1.0
            \task 9
            \task 9.0
        \end{tasks}

        % 14
        \item \lstinline!print(88 - 8)! 的运行结果是?(\qquad)
        \begin{tasks}(4)
            \task $88$
            \task $80$
            \task $88-8$
            \task $81$
        \end{tasks}
        
        \newpage
        % 15
        \item 分析下列程序,说法错误的是?(\qquad)
        \begin{lstlisting}
            import turtle
            turtle.color('blue')
            turtle.fillcolor('yellow')
            turtle.begin_fill()
            turtle.circle(50)
            turtle.end_fill()
            turtle.forward(100)
            turtle.color('red', 'aqua')
            turtle.begin_fill()
            turtle.circle(50)
            turtle.end_fill()
        \end{lstlisting}
        \begin{tasks}
            \task turtle.color('blue')表示的含义为:设置轮廓和填充颜色均为"blue"
            \task turtle.fllcolor('yellow')表示的含义为:设置填充颜色为"yellow"
            \task 程序运行结果为:绘制两个圆,左边圆填充颜色为"yellow",右边圆的颜色为"aqua"
            \task 最终绘制两个圆的轮廓颜色均为"blue"
        \end{tasks}

        % 16
        \item Python环境中,以下代码注释正确的是?(\qquad)
        \begin{tasks}(4)
            \task \lstinline!\#! 这个是一个程序
            \task \lstinline!/这个是一个程序/!
            \task \lstinline!"这是一个程序'!
            \task \lstinline!?这是一个程序?!
        \end{tasks}

        % 17
        \item \lstinline!print(5 % 10 + 5)! 的输出结果是?(\qquad)
        \begin{tasks}(4)
            \task 10
            \task 1/3
            \task 5.2
            \task 5
        \end{tasks}

        % 18
        \item 下列哪一个函数可以将海龟顺时针旋转?(\qquad)
        \begin{tasks}(4)
            \task \lstinline{left()}
            \task \lstinline{right()}
            \task \lstinline{back()}
            \task \lstinline{forward()}
        \end{tasks}

        % 19
        \item 在Python编程环境下,IDLE代表什么?(\qquad)
        \begin{tasks}(4)
            \task 编辑器
            \task 编译器
            \task 计算器
            \task 集成开发环境
        \end{tasks}

        % 20
        \item 如果某年的第1天也就是一月一日是星期一。星期一记作1,星期二记作2,以此类推,星期日记作0。要求这一年的第d天是星期几,下列哪一种方法可以实现?(\qquad)
        \begin{tasks}(4)
            \task \lstinline!d\%7!
            \task \lstinline!(d-1)\%7!
            \task \lstinline!(d-1)\%7+1!
            \task \lstinline!(d+1)\%7!
        \end{tasks}

        \newpage
        % 21
        \item 在初始状态下,执行以下命令后,turtle的坐标为?(\qquad)
        \begin{lstlisting}
            turtle.forward(10)
            turtle.left(90)
            turtle.forward(20)
        \end{lstlisting}
        \begin{tasks}(4)
            \task $(10,0)$
            \task $(10,20)$
            \task $(10,30)$
            \task $(10,-20)$
        \end{tasks}

        % 22
        \item 下列运算符中,哪一个不是比较运算符?(\qquad)
        \begin{tasks}(4)
            \task \lstinline{<}
            \task \lstinline{>}
            \task \lstinline{!=}
            \task \lstinline{=}
        \end{tasks}

        % 23
        \item 运行如下代码段,输出结果正确的是?(\qquad)
        \begin{lstlisting}
            word1  = "o"
            word2 = "n"
            print(word2 +word1)
        \end{lstlisting}
        \begin{tasks}(4)
            \task \lstinline{on}
            \task \lstinline{no}
            \task \lstinline{word3}
            \task \lstinline{word2word1}
        \end{tasks}

        % 24
        \item 下面哪一个不是Python的保留字?(\qquad)
        \begin{tasks}(4)
            \task class
            \task if
            \task turtle
            \task or
        \end{tasks}

        % 25
        \item 下面哪个代码可以绘制一个直径为200的填充为红色,轮廓为蓝边的圆形?(\qquad)
        \begin{tasks}(2)
            \task 
            \lstinline!import turtle! \\
            \lstinline!turtle.pencolor("blue")! \\
            \lstinline!turtle.fillcolor("red")! \\
            \lstinline!turtle.begin_fill()! \\
            \lstinline!turtle.circle(200)! \\
            \lstinline!turtle.end_fill()!

            \task
            \lstinline!import turtle! \\
            \lstinline!turtle.pencolor("blue")! \\
            \lstinline!turtle.fillcolor("red")! \\
            \lstinline!turtle.begin_fill()! \\
            \lstinline!turtle.circle(100,360)! \\
            \lstinline!turtle.end_fill()!

            \task 
            \lstinline!import turtle! \\
            \lstinline!turtle.color("blue")! \\
            \lstinline!turtle.dot(200)! 

            \task
            \lstinline!import turtle! \\
            \lstinline!turtle.pencolor("blue")! \\
            \lstinline!turtle.fillcolor("red")! \\
            \lstinline!turtle.dot(100)! 
        \end{tasks}
    \end{enumerate}

    {\noindent\heiti 第二部分、判断题(共 10 题,每题 2 分,共20分.)}
    \begin{enumerate}
        \setcounter{enumi}{25}
        % 26
        \item \lstinline{name = "John"},这个赋值语句书写正确(\qquad)

        %27
        \item 执行 \lstinline{turtle.hideturtle()} 命令隐藏海龟之后,再怎么移动也就不能在画布上画图了(\qquad)
        
        %28
        \item \lstinline!65 - 2 * 2 == 126! 运行结果为 \lstinline{True}.(\qquad)
  
        %29
        \item 一个字符串可以转化为任意数值(\qquad)
        
        %30
        \item 运行 \lstinline{turtle.clear()} 命令,将清空turtle窗口中的内容,turtle的位置会重置到窗口中央.(\qquad)
        
        %31
        \item Python中,“==”代表的是将左右两边的值进行比较,取平均值(\qquad)
        
        %32
        \item Word软件也可以用来编辑Python程序代码,也支持代码的调试和运行(\qquad)
        
        %33
        \item 在Python编程环境中,\lstinline!>>>! 提示符表示进入Python交互式命令行编程模式(\qquad)
        
        %34
        \item Python3中的单引号\lstinline{' '}和双引号 \lstinline{" "} 的作用是一样的(\qquad)
        
        %35
        \item 运行以下代码时会提示出错(\qquad)
        \begin{lstlisting}
            Python = "3.5.2"
            print(Python)
        \end{lstlisting}
    \end{enumerate}

    {\noindent\heiti 第三部分、编程题(共 2 题,共30分.)}
    \begin{enumerate}
        \setcounter{enumi}{35}
        
        % 36
        \item 求长方形的面积与周长.要求如下:
        \begin{tasks}[label=(\arabic*)]
            \task 程序开始运行后,输入长方形的长(a),然后再输入长方形的宽(b);
            \task 程序会根据输入的数字给出长方形的面积和长方形的周长;
            \task 输出长方形的面积和周长,并且注明是面积还是和周长.
        \end{tasks}
        \vfill

        %37
        \item 绘制如下图形,相关参数及要求如下:
        
        \begin{figure}[htbp]
            \begin{minipage}{.6\textwidth}
                \begin{tasks}[label=(\arabic*)]
                    \task 画笔起始位置不限,但是整个图形必须要能够在画布中呈现;
                    \task 大的正方形由四个小正方形组成;
                    \task 每个小正方形的边长均为100像素;
                    \task 左上角和右下角的正方形填充颜色均为黑色;
                    \task 所有的线条颜色均为黑色;
                    \task 绘制完成后,隐藏画笔.
                \end{tasks}
            \end{minipage}
            \begin{minipage}{.35\textwidth}
                \centering
                \includegraphics[width=.6\textwidth]{37.png}
            \end{minipage}
        \end{figure}
        
        \vfill
    \end{enumerate}
\end{document}