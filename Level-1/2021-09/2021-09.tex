\documentclass[11pt]{ctexart}

\usepackage{geometry}
\geometry{
    left = 0.6in,
    right = 0.6in,
    top = 0.8in,
    bottom = 1.0in
}
\usepackage{amssymb,amsbsy,amsmath,xcolor,mathrsfs,graphicx}
\usepackage{listings}
\usepackage{tasks}
\usepackage{pifont}
\settasks{
    label = \Alph*. ,
    label-width = 16pt
}

\renewcommand{\title}[3]{
    \begin{center}
        \Large\heiti 中国电子学会 #1~年~#2~月 Python~#3级考试
    \end{center}
}
\newcommand{\TimeAndName}[1]{
    \begin{center}
        考试时间:~#1~ 分钟 \qquad\qquad\qquad\qquad 姓名:\underline{\quad\quad\quad\quad}
    \end{center}
}

\begin{document}
    \lstset{
        language = python,
        keywordstyle = \color{orange}\bfseries,
        emph = {
            abs, all, any, ascii, bin, bool, breakpoint, bytearray, bytes,
            callable, chr, classmethod, compile, complex, copyright, credits,
            delattr, dict, dir, divmod, enumerate, eval, exec, exit, filter,
            float, format, frozenset, getattr, globals, hasattr, hash,
            help, hex, id, input, int, isinstance, issubclass, iter, len,
            license, list, locals, map, max, memoryview, min, next, object,
            oct, open, ord, pow, print, property, quit, range, repr, reversed,
            round, set, setattr, slice, sorted, staticmethod, str, sum, super,
            tuple, type, vars, zip,
        },
        emphstyle = \color{purple}\bfseries,
        showspaces = false,
        basicstyle = \ttfamily,
        morekeywords = {True,False}
    }

    \title{2021}{9}{一}
    
    \TimeAndName{60}
    
    {\noindent\heiti 第一部分、单选题(共 25 题,每题 2 分,共50分.)}

    \begin{enumerate}
        % 1
        \item 取整除的运算符是?(\qquad)
        \begin{tasks}(4)
            \task /
            \task //
            \task $\div$
            \task $**$
        \end{tasks}
    
        % 2
        \item 下面的程序为海龟绘制正方形的程序,请选择正确的选项将程序补全?(\qquad)
        \begin{lstlisting}
            import turtle
            turtle.forward(100)
            turtle.left(90)
            turtle.forward(100)
            turtle.left(90)
            turtle.forward(100)
            turtle.left(90)
            turtle.forward(    )
            turtle.left(90)
        \end{lstlisting}
        \begin{tasks}(4)
            \task $90$
            \task $-90$
            \task $0$
            \task $100$
        \end{tasks}

        % 3
        \item 已知\lstinline{a = 5}, \lstinline{a *= 2}, 那么 \lstinline!print(a)! 的结果为?(\qquad)
        \begin{tasks}(4)
            \task 5
            \task 2
            \task 10
            \task 20
        \end{tasks}
    
        % 4
        \item Python程序保存后的扩展名是?(\qquad)
        \begin{tasks}(4)
            \task .pyt
            \task .py
            \task .pn
            \task .ph
        \end{tasks}

        % 5
        \item 关于turtle,以下描述错误的是?(\qquad)
        \begin{tasks}(2)
            \task turtle中的画笔不能设置不同的形状
            \task turtle中的画笔可以设置移动的速度
            \task turtle中的画笔可以设置不同的颜色
            \task turtle中的画笔可以设置不同的大小
        \end{tasks}

        % 6
        \item 运行以下代码的结果是?(\qquad)
        \begin{lstlisting}
            print(2021<=2020 or 2022>2018)
        \end{lstlisting}
        \begin{tasks}(4)
            \task False
            \task True
            \task 20212018
            \task 2022>2018
        \end{tasks}

        % 7
        \item \lstinline!print("17+2")!输出的结果是?(\qquad)
        \begin{tasks}(4)
            \task \lstinline!"17+2"!
            \task 19
            \task 172
            \task 17+2
        \end{tasks}

        \newpage
        % 8
        \item \lstinline!turtle.goto(x,y)!的含义为下列选项的哪一个?(\qquad)
        \begin{tasks}
            \task 以目前坐标为原点,画一个边长为$x$和$y$的矩形
            \task 画笔提笔,移动到$x,y$的位置
            \task 按照现在画笔状态,将画笔移动到坐标为$x,y$的位置
            \task 将目前原点移动到 $x,y$的位置 
        \end{tasks}

        % 9
        \item \lstinline!turtle.circle(150, steps=5)!命令能绘制出以下哪个图形?(\qquad)
        \begin{tasks}
            \task 直径(从顶点到图形中心的距离的2倍)为150像素的圆内接正五边形
            \task 半径(从顶点到图形的中心)为150像素的圆内接正五边形
            \task 半径(从顶点到图形的中心)为150像素的圆内接五角星
            \task 边长为150像素的正五边形
        \end{tasks}

        % 10
        \item 关于Python的编程环境,下列表述是错误的是?(\qquad)
        \begin{tasks}(2)
            \task Python有多种编程环境
            \task Python自带的编程环境是IDLE
            \task Python的编程是纯图形化的
            \task Python可以导入多个第三方库
        \end{tasks}

        % 11
        \item 下面哪一条命令是用来定义画笔宽度的?(\qquad)
        \begin{tasks}(4)
            \task \lstinline!turtle.pencolor()!
            \task \lstinline!turtle.speed()!
            \task \lstinline!turtle.pensize()!
            \task \lstinline!turtle.shape()!
        \end{tasks}

        % 12
        \item 请观察以下数字的规律:$2+6,3+8,4+10,(\quad),6+14$,请问括号中应该填写什么?(\qquad)
        \begin{tasks}(4)
            \task 5+11
            \task 5+12
            \task 4+9
            \task 6+13
        \end{tasks}

        % 13
        \item  海龟绘图时,使用 \lstinline{speed(n)},当 \lstinline{n} 的值是多少时,绘图的速度最快?(\qquad)
        \begin{tasks}(4)
            \task 3
            \task 2
            \task 1
            \task 0
        \end{tasks}

        % 14
        \item \lstinline!print(51 % 10)! 输出的结果是?(\qquad)
        \begin{tasks}(4)
            \task 1
            \task 0
            \task True
            \task False
        \end{tasks}
        
        % 15
        \item 下列哪个函数的功能是进行输出?(\qquad)
        \begin{tasks}(4)
            \task print()
            \task input()
            \task get()
            \task range()
        \end{tasks}

        % 16
        \item \lstinline!print(4+6*2+8)!的结果是?(\qquad)
        \begin{tasks}(4)
            \task 100
            \task 24
            \task 64
            \task 28
        \end{tasks}

        % 17
        \item 下面\lstinline!print()!语句,哪一个是正确的用法?(\qquad)
        \begin{tasks}(4)
            \task \lstinline{print"(welcome!)"}
            \task \lstinline{print("welcome"!)}
            \task \lstinline{print("welcome!')}
            \task \lstinline{print("welcome!")}
        \end{tasks}

        % 18
        \item \lstinline{turtle} 回到原点的命令是?(\qquad)
        \begin{tasks}(4)
            \task \lstinline{hideturtle()}
            \task \lstinline{clear()}
            \task \lstinline{circle()}
            \task \lstinline{home()}
        \end{tasks}

        % 19
        \item 逻辑运算符中(注意不是所有的运算符,只是逻辑运算符),优先级最高的是?(\qquad)
        \begin{tasks}(4)
            \task or
            \task and
            \task not
            \task $**$
        \end{tasks}

        % 20
        \item 观察下面的程序,运行结果为?(\qquad)
        \begin{lstlisting}
            a = 8
            a += 1
            print(a)
            a *= 10
            print(a)
            a **= 2
            print(a)
        \end{lstlisting}
        \begin{tasks}(4)
            \task 9
            \task 90
            \task 8100
            \task 9 \\ 90 \\ 8100
        \end{tasks}

        % 21
        \item 运行以下代码的结果是?(\qquad)
        \begin{lstlisting}
            a = "Python2"
            b = "Python3"
            c=a+b
            print(c)
        \end{lstlisting}
        \begin{tasks}(4)
            \task 5
            \task Python5
            \task c
            \task Python2Python3
        \end{tasks}

        % 22
        \item 下面代码运行的结果是?(\qquad)
        \begin{lstlisting}
            a = 1.27
            print(eval('a+10'))
        \end{lstlisting}
        \begin{tasks}(4)
            \task 1.2710
            \task 系统报错
            \task 11.27
            \task 1.27+10
        \end{tasks}

        % 23
        \item 使用哪个函数可以把字符串\lstinline!'123'!转换为整型123?(\qquad)
        \begin{tasks}(4)
            \task num()
            \task str()
            \task float()
            \task int()
        \end{tasks}

        \newpage
        % 24
        \item 关于变量的说法,错误的是?(\qquad)
        \begin{tasks}(2)
            \task 变量必须要命名
            \task 变量第二次复制后,第一次赋的值将被删除
            \task 变量可以用来存储数字,也可以存储文字
            \task 在同一个程序里,变量名能重复
        \end{tasks}

        % 25
        \item 关于Python的编程环境,下列的哪个表述是不正确的?(\qquad)
        \begin{tasks}
            \task Python自带的编程环境是IDLE

            \task 下载安装好Python软件后,无需单独下载IDLE

            \task IDLE的交互编程环境中,可以一次写入多行无缩进的语句代码,然后进行运行 

            \task 为了保存编写的代码,我们通常使用IDLE中的脚本式编程模式 
        \end{tasks}
    \end{enumerate}

    {\noindent\heiti 第二部分、判断题(共 10 题,每题 2 分,共20分.)}
    \begin{enumerate}
        \setcounter{enumi}{25}
        % 26
        \item 运行下列Python代码后,a和b的数据类型均为字符串(\qquad)
        \begin{lstlisting}
            a = '5'
            b = "b"
        \end{lstlisting}

        %27
        \item 在turtle库中,画笔的起点在画布的正中央,\lstinline{turtle.goto()}、\lstinline{turtle.setx()} 和 \lstinline{turtle.sety()} 使用的都是基于中心点 $(0,0)$ 的绝对坐标(\qquad)
        
        %28
        \item \lstinline{turtle.shape("square")} 命令可以将turtle形状设置为海龟(\qquad)
  
        %29
        \item and 是Python中常用的保留字,不可以作为变量名(\qquad)
        
        %30
        \item Python的IDLE编程中有交互模式和脚本模式两种编程方式(\qquad)
        
        %31
        \item 程序:a=b中,a是变量,b是值(\qquad)
        
        %32
        \item from, False, import, as 是Python中常用保留字,不可以作为变量名(\qquad)
        
        %33
        \item \lstinline!input()!语句是用来输入一个指令(\qquad)
        
        %34
        \item 以下代码运行后显示的结果是3(\qquad)
        \begin{lstlisting}
            b = 3
            c = a + b
            print(c)
        \end{lstlisting}
        
        %35
        \item Python代码的注释只有一种方式,那就是使用 \lstinline{#} 符号(\qquad)
    \end{enumerate}

    \newpage
    {\noindent\heiti 第三部分、编程题(共 2 题,共30分.)}
    \begin{enumerate}
        \setcounter{enumi}{35}
        
        % 36
        \item 龟兔赛跑:
        
        龟兔赛跑,兔子刚开始跑得非常快,但是兔子太骄傲了,在领先乌龟 $100$ 米时,自行休息睡着了,乌龟一步一步进行追赶,乌龟的速度是 $V\ \mathrm{m/s}(V<10)$,请计算出乌龟多长时间就可以追上兔子呢?要求:
        \begin{tasks}[label=(\arabic*)]
            \task 程序开始运行时,询问请输入乌龟爬行的速度,输入一个数字;
            \task 程序根据输入的数字计算出乌龟需要多长时间就可以追上兔子;
            \task 输出的格式为:“乌龟能够追上兔子所需的时间是:XX秒。”.(XX表示计算的结果)
        \end{tasks}
        \vfill

        %37
        \item 绘制如下图形:
        
        \begin{tasks}[label=(\arabic*)]
            \task 画一个由一个正方形和一个菱形组成的图形,其中,正方形的边长为200像素,菱形的四个顶点均在正方形四条边的中点上;
            \task 设置画笔速度为1;
            \task 菱形的填充颜色为红色,所有线条为黑色;
            \task 画图结束,隐藏并停止画笔.
        \end{tasks}
        \begin{center}
            \includegraphics[width=.1\textwidth]{37.png}
        \end{center}
        \vfill
    \end{enumerate}
\end{document}